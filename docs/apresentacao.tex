% !TEX TS-program = pdflatex
% !TEX encoding = UTF-8 Unicode

\documentclass[portuges]{beamer}

\definecolor{darkred}{rgb}{0.5,0,0}

\mode<presentation>
{
  \usetheme{Warsaw}
  %\usecolortheme{seahorse}
  \setbeamercolor*{palette primary}{use=structure,fg=white,bg=darkred}
  \setbeamercolor*{section in toc}{fg=darkred}
  \setbeamercolor*{palette quaternary}{fg=white,bg=gray!30!black}
  \setbeamercolor*{item}{fg=red}
%  \setbeamercovered{transparent}
}

\usepackage[portuges]{babel}

\usepackage[utf8]{inputenc}

\usepackage{times}
\usepackage[T1]{fontenc}

\title
{MyStreet}

\subtitle
{Rede social para parttilha de informação sobre infraestruturas}

\author[Li4]
{
Bruno Matos nº 33147\\ 
Carlos Cosio nº 5177\\
Jorge Rodrigues nº 51751\\
Miguel Fernandes nº 44024\\ 
Rui Silva nº 47082\\
}

\date % [Short Occasion] % (optional)
{LI4 2012-2013}

\subject{MyStreet Apresentação}
% This is only inserted into the PDF information catalog. Can be left
% out. 

% Logo
%\pgfdeclareimage[height=0.5cm]{university-logo}{LogoUM.jpg}
%{\pgfuseimage{university-logo}}
\logo{
   \includegraphics[height=0.5cm]{LogoUM.jpg}
   }

% Delete this, if you do not want the table of contents to pop up at
% the beginning of each subsection:
%\AtBeginSubsection[]
%{
%  \begin{frame}<beamer>{}
 %   \tableofcontents[currentsection,currentsubsection]
 % \end{frame}
%}


% If you wish to uncover everything in a step-wise fashion, uncomment
% the following command: 

\beamerdefaultoverlayspecification{<+->}

\begin{document}

\begin{frame}
  \titlepage
\end{frame}

\begin{frame}{Visão Geral}
  \tableofcontents
  % You might wish to add the option [pausesections]
\end{frame}


% Since this a solution template for a generic talk, very little can
% be said about how it should be structured. However, the talk length
% of between 15min and 45min and the theme suggest that you stick to
% the following rules:  

% - Exactly two or three sections (other than the summary).
% - At *most* three subsections per section.
% - Talk about 30s to 2min per frame. So there should be between about
%   15 and 30 frames, all told.

\section{Introdução}

\subsection{Conceito}

\begin{frame}{Em que consiste}%{Subtitles are optional.}
  % - A title should summarize the slide in an understandable fashion
  %   for anyone how does not follow everything on the slide itself.

  \begin{itemize}
  \item Troca de informação sobre infraestrturas
    %Use \texttt{itemize} a lot.
  \item Relato de problemas
   % Use very short sentences or short phrases.
  \item Atualizações do estado dos problemas
  \item Sugestões de resolução
  \end{itemize}
\end{frame}

\begin{frame}{Make Titles Informative.}

  You can create overlays\dots
  \begin{itemize}
  \item using the \texttt{pause} command:
    \begin{itemize}
    \item
      First item.
      \pause
    \item    
      Second item.
    \end{itemize}
  \item
    using overlay specifications:
    \begin{itemize}
    \item<3->
      First item.
    \item<4->
      Second item.
    \end{itemize}
  \item
    using the general \texttt{uncover} command:
    \begin{itemize}
      \uncover<5->{\item
        First item.}
      \uncover<6->{\item
        Second item.}
    \end{itemize}
  \end{itemize}
\end{frame}


\subsection{Second Subsection}

\begin{frame}{Make Titles Informative.}
\end{frame}

\begin{frame}{Make Titles Informative.}
\end{frame}



\section*{Summary}

\begin{frame}{Summary}

  % Keep the summary *very short*.
  \begin{itemize}
  \item
    The \alert{first main message} of your talk in one or two lines.
  \item
    The \alert{second main message} of your talk in one or two lines.
  \item
    Perhaps a \alert{third message}, but not more than that.
  \end{itemize}
  
  % The following outlook is optional.
  \vskip0pt plus.5fill
  \begin{itemize}
  \item
    Outlook
    \begin{itemize}
    \item
      Something you haven't solved.
    \item
      Something else you haven't solved.
    \end{itemize}
  \end{itemize}
\end{frame}


\end{document}


